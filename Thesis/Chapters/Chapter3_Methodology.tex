\chapter{ METHODOLOGY} 
\label{chapter3} 
This chapter outlines the systematic approach adopted for the research, detailing each step from development of a new OpenFOAM based CFD-DEM solver to validation through comparative analysis with existing solvers. Below is a flowchart summarizing the methodology framework employed in this study:\par

\begin{figure}[ht]
	\centering
	\includegraphics[width=1.0\textwidth]{Figures/Chapters/C03/methodology_chart.png}
	\caption{Methodology Flowchart}
	\label{fig:Methodology Flowchart}
\end{figure}

\section{Development of a CFD-DEM Solver in OpenFOAM}

The development of the CFD-DEM solver involved a systematic process of environment setup, codebase study, and modular implementation of DEM physics and coupling algorithms.

\subsection{Environment Setup: C++ in WSL and OpenFOAMv2312}
The development environment was established using the Windows Subsystem for Linux (WSL), providing a Linux-based development platform on a Windows host. OpenFOAM-v2312 from OpenCFD Ltd. is forked from url \url{https://develop.openfoam.com/Development/openfoam/-/blob/master/doc/Build.md} as the core CFD framework. The source code of the OpenFOAM-v2312 was compiled with g++ flags \textit{-O0 -DFULLDEBUG} to allow runtime debugging of the code. Visual Studio Code (VS Code) was utilized as the primary Integrated Development Environment (IDE), configured with C++ extensions for code navigation, debugging, and compilation management.

\subsection{Study of FOAM Classes and Solvers}
The OpenFOAM class hierarchy was studied in detail to understand the underlying architecture. Key classes analyzed included:
\begin{itemize}
	\item \textbf{fvMesh:} For handling computational grid and topology.
	\item \textbf{volScalarField/volVectorField:} For managing field variables like pressure and velocity.
	\item \textbf{pimpleFoam:} For understanding the PIMPLE algorithm for transient, incompressible flow, which served as base for the fluid solver.
\end{itemize}

\subsection{DEM Implementation}
The Discrete Element Method (DEM) was implemented as a standalone library within the `bashyal/Src/dem` directory. The approach focused on modularity and extensibility:
\begin{itemize}
	\item \textbf{Particle Class:} A custom `particle` class was created to store properties (mass, position, velocity) and handle motion integration using Newton's laws.
	\item \textbf{Collision Detection:} The Gilbert-Johnson-Keerthi (GJK) algorithm was implemented for robust collision detection between arbitrary convex shapes (cubes).
	\item \textbf{Contact Models:} The Walton-Braun hysteretic spring model was integrated to calculate normal forces, accounting for energy dissipation, along with a Coulomb friction model for tangential forces.
\end{itemize}

\subsection{CFD-DEM Coupling}
The coupling between the fluid and solid phases was achieved by developing a new solver, `fineDEMfoam`. The coupling strategy involved:
\begin{itemize}
	\item \textbf{Solver Structure:} The solver integrates the standard `pimpleFoam` loop with the custom DEM library.
	\item \textbf{Immersed Boundary Method (IBM):} A fictitious domain approach was used where particles are mapped onto the fluid grid using a solid volume fraction field ($\epsilon$).
	\item \textbf{Force Calculation:} Fluid forces (buoyancy, drag) are calculated based on the pressure gradient and interpolated velocity fields, then applied to the particles.
	\item \textbf{Volume of Fluid (VOF):} The VOF method was retained to capture the free surface, allowing for the simulation of three-phase flows (air-water-sediment).
\end{itemize}

\section{Solver Validation and Comparative Analysis}

This section details the validation of the developed solver through fundamental test cases and comparative analysis with existing solvers.

\subsection{Fundamental Validation Test Cases}

To validate the developed solver, four fundamental test cases are established. These cases isolate specific physics: mechanical contact (DEM) and fluid-structure interaction (CFD-DEM).

\subsubsection{Test Case 1: DEM Validation - The Hysteretic Bounce}
This test validates the GJK collision detection, Time Integration, and specifically the Walton-Braun Hysteretic force model.

\textbf{Theory:}
In a hysteretic model, energy is dissipated by the difference in stiffness during loading ($k_1$) and unloading ($k_2$). The effective Coefficient of Restitution ($e$) is derived analytically as:
\begin{equation}
	e = \sqrt{\frac{k_1}{k_2}}
\end{equation}

\textbf{Simulation Parameters:}
\begin{itemize}
	\item \textbf{Geometry:} A single cube of side length $L=0.1\,m$ is dropped from a height $H_0=1.0\,m$ onto a fixed floor at $z=0$. The initial centroid position is $z = 1.05\,m$.
	\item \textbf{Particle Properties:} Mass $m=1.0\,kg$.
	\item \textbf{Contact Model:} Walton-Braun Hysteretic Spring.
	\begin{itemize}
		\item Loading Stiffness: $k_1 = 1 \times 10^5 \, N/m$.
		\item Unloading Stiffness: $k_2 = 4 \times 10^5 \, N/m$.
		\item Coefficient of Restitution: $e = \sqrt{k_1/k_2} = 0.5$.
		\item Friction: $\mu = 0.0$.
	\end{itemize}
	\item \textbf{Controls:} Time step $dt = 1 \times 10^{-5}\,s$, End time $= 1.5\,s$, Gravity $g=9.81\,m/s^2$.
\end{itemize}

\begin{figure}[H]
	\centering
	\includegraphics[width=0.5\textwidth]{Figures/Chapters/C03/setup_bounce.png}
	\caption{Simulation setup for Test Case 1: Hysteretic Bounce showing the initial drop height of the cubic particle.}
	\label{fig:setup_bounce}
\end{figure}

\subsubsection{Test Case 2: DEM Validation - Sliding Block on an Inclined Plane}
This test validates the Tangential Force Model, Coulomb Friction Limit, and Gravity Projection.

\textbf{Theory:}
For a block on an incline $\theta$:
\begin{itemize}
	\item Driving Force: $F_{down} = m g \sin(\theta)$
	\item Resistive Force: $F_{fric} = \mu F_{normal} = \mu m g \cos(\theta)$
	\item Net Acceleration: $a = g (\sin\theta - \mu \cos\theta)$
\end{itemize}

\textbf{Simulation Parameters:}
\begin{itemize}
	\item \textbf{Geometry:} A fixed rectangular floor tilted at $\theta = 30^\circ$. The particle is a cube with side length $L=0.1\,m$ and mass $m=1.0\,kg$.
	\item \textbf{Contact Model:} Linear Spring-Dashpot with Friction.
	\begin{itemize}
		\item Normal Stiffness: $k_n = 1 \times 10^4 \, N/m$.
		\item Tangential Stiffness: $k_t = 1 \times 10^4 \, N/m$.
		\item Friction Coefficient: $\mu = 0.3$.
		\item Damping: $\gamma_N = 200.0$.
	\end{itemize}
	\item \textbf{Controls:} Time step $dt = 1 \times 10^{-4}\,s$, End time $= 1.0\,s$.
\end{itemize}

\begin{figure}[H]
	\centering
	\includegraphics[width=0.6\textwidth]{Figures/Chapters/C03/setup_sliding.png}
	\caption{Simulation setup for Test Case 2: Sliding Block on a $30^\circ$ inclined plane.}
	\label{fig:setup_sliding}
\end{figure}

\subsubsection{Test Case 3: DEM Validation - Random Loose Packing}
This test validates multi-body contact stability, geometric exclusion, and energy dissipation in a dense granular system.

\textbf{Theory:}
Random loose packing of frictional cubes is expected to yield a solid fraction $\phi \approx 0.50$. The system settles when kinetic energy dissipates via inelastic collisions and friction.

\textbf{Simulation Parameters:}
\begin{itemize}
	\item \textbf{Geometry:} A container with a $0.5\,m \times 0.5\,m$ base ($10L \times 10L$).
	\item \textbf{Particles:} 200 Cubes with side length $L=0.05\,m$ and mass $m=1.0\,kg$ each.
	\item \textbf{Contact Model:} Linear Spring-Dashpot with Friction.
	\begin{itemize}
		\item Stiffness: $k = 1 \times 10^5 \, N/m$.
		\item Friction Coefficient: $\mu = 0.5$.
		\item Damping: $\gamma_N = 50.0$.
	\end{itemize}
	\item \textbf{Controls:} Time step $dt = 1 \times 10^{-4}\,s$, End time $= 5.0\,s$.
\end{itemize}

\begin{figure}[H]
	\centering
	\includegraphics[width=0.6\textwidth]{Figures/Chapters/C03/setup_packing.png}
	\caption{Simulation setup for Test Case 3: Random Loose Packing of 200 cubic particles.}
	\label{fig:setup_packing}
\end{figure}

\subsubsection{Test Case 4: CFD-DEM Validation - Free Sedimentation of a settling cube}
This test case evaluates the performance of the solver in simulating the free fall of a single settling cube in a quiescent fluid. The particle motion is governed by the balance between gravitational, buoyancy, and hydrodynamic drag forces.

\textbf{Theory:}
A particle falling in a fluid accelerates until Drag + Buoyancy equals Gravity.
\begin{equation}
	F_g = F_b + F_d \implies m g = (\rho_f V_{disp} g) + \left( \frac{1}{2} \rho_f v_t^2 C_d A \right)
\end{equation}

\textbf{Domain and Mesh:}
\begin{itemize}
	\item \textbf{Dimensions:} A vertical water column of $0.02\,m \times 1.0\,m \times 0.02\,m$.
	\item \textbf{Mesh Resolution:} $40 \times 2000 \times 40$ cells, ensuring high vertical resolution to capture the settling wake.
\end{itemize}

\begin{figure}[H]
	\centering
	\includegraphics[width=0.5\textwidth]{Figures/Chapters/C03/single_particle_case.png}
	\caption{Geometry for simulating free sedimentation of a single settling cube with dimensions in cm}
	\label{fig:Geometry for simulating free sedimentation of a single settling cube with dimensions in cm}
\end{figure}

\textbf{Particle Properties:}
\begin{itemize}
	\item \textbf{Geometry:} Cube ($L = 2\,mm$ implied from STL).
	\item \textbf{Initial Position:} $(0.01, 0.9, 0.01)\,m$.
	\item \textbf{Material:} Density $\rho_s = 2000\,kg/m^3$, Young's Modulus $Y=1 \times 10^7\,Pa$, Poisson Ratio $\nu=0.3$.
	\item \textbf{Interaction:} Friction $\mu=0.5$, Restitution $\epsilon=0.8$.
\end{itemize}

\textbf{Simulation Controls:}
\begin{itemize}
	\item Fluid Time Step: $\Delta t = 5 \times 10^{-4}\,s$.
	\item DEM Sub-cycling: Matched to fluid step ($5 \times 10^{-4}\,s$).
	\item End Time: $4.0\,s$.
\end{itemize}

\begin{figure}[H]
	\centering
	\includegraphics[width=0.5\textwidth]{Figures/Chapters/C03/test_case_mesh.png}
	\caption{Immersed Boundary Mesh representing sediment particle of size 2 mm positioned near the top of water column}
	\label{fig:Immersed Boundary Mesh representing sediment particle of size 2 mm positioned near the top of water column}
\end{figure}

\subsubsection{Test Case 5: CFD-DEM Validation - The Floating Cube Equilibrium}
This test validates the VOF-IBM Coupling and Free-Surface Intersection.

\textbf{Theory:}
Archimedes' principle for a floating object:
\begin{equation}
	\rho_s L^3 g = \rho_f (L^2 h_{sub}) g
\end{equation}

\textbf{Domain and Mesh:}
\begin{itemize}
	\item \textbf{Dimensions:} $0.3\,m \times 1.0\,m \times 0.3\,m$.
	\item \textbf{Mesh Resolution:} $30 \times 100 \times 30$ cells.
\end{itemize}

\textbf{Simulation Parameters:}
\begin{itemize}
	\item \textbf{Particle:} Cube ($L=0.1\,m$) initialized at $(0.15, 0.6, 0.15)\,m$.
	\item \textbf{Material:} Density $\rho_s = 500\,kg/m^3$ (to ensure floating), with other mechanical properties identical to Test Case 4.
	\item \textbf{Controls:} Time Step $\Delta t = 1 \times 10^{-4}\,s$, End Time $= 4.0\,s$.
\end{itemize}

%\begin{figure}[ht]
%	\centering
%	\includegraphics[width=0.5\textwidth]{Figures/Chapters/C03/setup_floating.png}
%	\caption{Setup for Test Case 5: A cube interacting with the air-water free surface.}
%	\label{fig:setup_floating}
%\end{figure}

\section{Primary Application Study Case}
The primary study case involves the simulation of turbulent free-surface flow over a mobile sediment bed in a rectangular channel. This case evaluates the solver's capability to simulate realistic transport regimes (bedload and suspension) under varying flow conditions using different RANS turbulence closures.
\par

\subsubsection{Test Case A: Particle-Laden Flow in a Rectangular Channel}

To evaluate the performance of the newly developed CFD-DEM solver in handling complex particle-fluid interactions, a test case involving multiple particles in a rectangular channel flow is simulated. This configuration tests the solver's capability to handle particle-particle collisions, particle-wall interactions, and the coupling between fluid flow and particle motion. 

\subsubsection{Simulation Regimes}
Two distinct flow regimes were simulated to observe different transport mechanisms:

\begin{table}[ht]
	\centering
	\caption{Flow regimes and inlet conditions for Study Case A}
	\label{tab:case_a_regimes}
	\begin{tabular}{|l|c|c|l|}
		\hline
		\textbf{Case Name} & \textbf{Slope} & \textbf{Inlet Velocity ($U_{inlet}$)} & \textbf{Flow Regime} \\ \hline
		Sub-Critical & 0.01 & 1.0 m/s & Bedload dominance \\ \hline
		Super-Critical & 0.02 & 1.5 m/s & Suspension/Saltation \\ \hline
	\end{tabular}
\end{table}

\subsubsection{Domain and Particles}
The computational domain and granular bed properties were consistent across all simulations:
\begin{itemize}
	\item \textbf{Domain Dimensions:} Length $L=2.0\,m$, Width $W=0.1\,m$, Height $H=0.4\,m$.
	\item \textbf{Grid Resolution:} $400 \times 20 \times 80$ cells ($dx=dy=0.005\,m$).
	\item \textbf{Initial Water Depth:} $h_{water} = 0.2\,m$.
	\item \textbf{Sediment Bed:} A randomly packed bed of approximately 60 particles (size $2\,cm$, density $2500\,kg/m^3$) initialized between $x=0.5\,m$ and $x=1.5\,m$.
\end{itemize}

\subsubsection{Boundary Conditions}
\begin{itemize}
	\item \textbf{Inlet ($x=0$):} 
	\begin{itemize}
		\item Phase fraction $\alpha=1$ (water) for $y \le h_{inlet}$.
		\item Velocity fixed at $U_{inlet}$ for the water phase.
		\item Turbulent kinetic energy $k_{inlet} = 1.5 (I \cdot U_{inlet})^2$ with intensity $I=0.05$.
		\item Dissipation ($\epsilon$ or $\omega$) derived from a mixing length $L_{scale} = 0.07 h_{inlet}$.
	\end{itemize}
	\item \textbf{Outlet ($x=L$):} Zero gradient for velocity and phase; fixed hydrostatic pressure profile.
	\item \textbf{Walls:} The bottom boundary ($y=0$) is a no-slip wall with standard wall functions. The top boundary ($y=H$) is treated as a free slip (atmosphere) condition.
\end{itemize}

%\begin{figure}[ht]
%	\centering
%	\includegraphics[width=0.7\textwidth]{Figures/Chapters/C03/sediment_bed_sim.png}
%	\caption{Geometry for simulating particle-laden flow in a rectangular channel with dimensions in cm}
%	\label{fig:Geometry for simulating particle-laden flow in a rectangular channel with dimensions in cm}
%\end{figure}



