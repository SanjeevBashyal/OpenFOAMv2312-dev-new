\chapter{ CONCLUSIONS AND OUTLOOK}
\label{chapter6}

\section{Summary of Research}

This research presented the development, implementation, and validation of \textbf{fineDEMfoam}, a coupled CFD-DEM solver within the OpenFOAM framework designed to simulate sediment transport in free surface turbulent flows. The solver couples the Finite Volume Method (FVM) for the fluid phase, governed by the Reynolds-Averaged Navier-Stokes (RANS) equations, with the Discrete Element Method (DEM) for the solid phase. A key feature of the implementation is the utilization of a Immersed Boundary Method (IBM) to resolve fluid forces on polygonal particles, alongside the Gilbert-Johnson-Keerthi (GJK) algorithm for collision detection. The research systematically progressed from fundamental verification of contact mechanics to complex coupled simulations involving sediment particles in open channel systems.

\section{Fulfillment of Objectives}

The extent to which the specific research objectives outlined in Chapter 1 were achieved is detailed below:


\section{Fulfillment of Objectives}

The extent to which the specific research objectives outlined in Chapter 1 were achieved is detailed below:

\begin{itemize}
	\item \textbf{Objective 1: To develop a mathematical model that combines RANS turbulence closure models with discrete element method to simulate particle transport processes during free surface water flows.}
	\par This objective was achieved through the development and implementation of the `fineDEMfoam` solver. The mathematical model couples the Reynolds-Averaged Navier-Stokes (RANS) equations for the fluid phase with the equations of motion for individual solid particles, solved using the Discrete Element Method (DEM). To handle free surface flows, the Volume of Fluid (VOF) method was integrated to track the air-water interface. The coupling between the Eulerian fluid grid and the Lagrangian particles was realized using an Immersed Boundary Method (IBM), which accounts for the momentum exchange and solid volume fraction, thus creating a comprehensive model for simulating sediment transport in turbulent, free-surface environments.
	
	\item \textbf{Objective 2: To implement the model in OpenFOAM framework and to assess the predictive capabilities of the model.}
	\par The mathematical model was successfully implemented within the OpenFOAM framework. The solver's predictive capabilities were rigorously assessed through a series of validation test cases. The mechanical accuracy of the DEM module was confirmed via the hysteretic bounce test and the sliding block test. The CFD-DEM coupling was validated against the experimental correlations of \textcite{dietrich_settling_1982} for free sedimentation, demonstrating close agreement in terminal velocity predictions. Furthermore, the solver's ability to handle three-phase (Air-Water-Sediment) interactions was confirmed by the floating cube test case, which validated the coupling of the IBM with the VOF method.
	
	\item \textbf{Objective 3: To identify the most suitable two equation standard RANS turbulence closure scheme for the solver for better prediction of sediment transport phenomenon near bed region in open channel flow.}
	\par This objective was addressed through a comparative study involving three standard two-equation RANS models: $k-\epsilon$, Standard $k-\omega$, and $k-\omega$ SST, applied to sediment transport in an open channel. The study found that the $k-\epsilon$ model tended to over-predict sediment flux due to excessive turbulence production at particle stagnation points, leading to premature erosion. The Standard $k-\omega$ model proved overly dissipative, under-predicting transport activity. The $k-\omega$ SST model was identified as the most suitable closure scheme, as it provided a physically realistic balance by accurately capturing turbulence in particle wakes and bed-shielding effects, resulting in better predictions of sediment transport dynamics near the bed.
\end{itemize}

\section{Key Conclusions}

Based on the results and discussion, the following conclusions are drawn:

\begin{enumerate}
	\item A CFD-DEM model for free-surface sediment transport was developed by coupling RANS turbulence models with a DEM via an IBM, and by integrating the VoF method to resolve the air-water interface.
	\item The solver demonstrates strong predictive capabilities, showing good agreement with fundamental physics and experimental data (Dietrich, 1982) in the test cases. It successfully simulates diverse scenarios, from quiescent settling and free-surface interaction to saltation in open-channel flow.
	\item For CFD-DEM applications involving free surface flow, the \(k-\omega\) SST model is recommended as it predicts high TKE production in the particle wakes without forcing excessive particle velocity.
\end{enumerate}

\section{Limitations}

While the developed solver represents a significant advancement, several limitations remain:

\begin{itemize}
	\item \textbf{RANS Averaging:} The use of RANS models filters out instantaneous turbulent fluctuations which are known to contribute to the incipient motion of sediment particles.
	\item \textbf{Mesh Resolution Dependence:} The accuracy of the IBM is dependent on the ratio of the fluid cell size to the particle diameter. High-resolution grids are required for accurate drag calculation which significantly increase computational cost.
	\item \textbf{Particle Shape Simplification:} Although the solver handles cubes and convex polyhedra, the current validation was primarily focused on cubic geometries. The behavior of highly irregular, concave natural sediment grains remains to be fully characterized.
\end{itemize}

\section{Recommendations for Future Work}

To further enhance the capabilities and accuracy of the `fineDEMfoam` solver, the following future work is recommended:

\begin{itemize}
	\item \textbf{Integration with LES:} Replacing the RANS turbulence framework with Large Eddy Simulation (LES) would allow the solver to capture instantaneous turbulent structures and their direct impact on particle entrainment.
	\item \textbf{Concave and Natural Geometries:} Future studies should incorporate geometries of river aggregates (involving curved surfaces and concave shapes) to evaluate the impact of natural shape variations on packing density and hydraulic resistance.
	\item \textbf{Parallelization:} To enable reach-scale simulations, the particle-mesh interaction algorithms should be further optimized for massive parallelization either with MPI or with GPUs to handle thousands of particles efficiently.
\end{itemize}