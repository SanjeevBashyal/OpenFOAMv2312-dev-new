\chapter{ METHODOLOGY} 
\label{chapter4} 
This chapter outlines the systematic approach adopted for the research, detailing each step from development of a new OpenFOAM based CFD-DEM solver to validation through comparative analysis with existing solvers. Below is a flowchart summarizing the methodology framework employed in this study:\par

\begin{figure}[ht]
	\centering
	\includegraphics[width=1.0\textwidth]{Figures/Chapters/C04/methodology_chart.png}
	\caption{Methodology Flowchart}
	\label{fig:Methodology Flowchart}
\end{figure}

\section{Solver Validation and Comparative Analysis}

To validate the performance and accuracy of the newly developed OpenFOAM based CFD-DEM solver, simulations are conducted on three fundamental test cases and the results are compared with those obtained from two state-of-the-art Eulerian-Lagrangian coupling frameworks. The CFDEM{\circledR} coupling framework \parencite{hager_parallel_2014} integrates OpenFOAM with LIGGGHTS{\circledR} DEM engine, providing a robust platform for resolved and unresolved particle-fluid interactions. The OpenHFDIB-DEM solver \parencite{studenik_openhfdib-dem_2024} extends OpenFOAM with a hybrid Fictitious Domain-Immersed Boundary method, enabling simulations with arbitrarily shaped particles. Three test cases are considered: free sedimentation of a single spherical particle, DEM pouring heap formation, and particle-laden flow in a channel, which are fundamental validation benchmarks for CFD-DEM solvers. These cases are selected due to their relevance to sediment transport processes and their ability to validate the accuracy of particle-particle interactions and particle-fluid coupling mechanisms. For each test case, simulations are performed using the newly developed solver, and the results are systematically compared with those obtained from CFDEM Coupling and OpenHFDIB-DEM solvers. The simulations are conducted under controlled configurations, with initial and boundary conditions carefully set to replicate experimental scenarios, ensuring a fair and meaningful comparison across all solvers.\par

\subsection{Test Case A: Free Sedimentation of a Single Spherical Particle}

This test case evaluates the performance of the newly developed CFD-DEM solver in simulating the free fall of a single spherical particle in a quiescent fluid, with results compared against CFDEM Coupling and theoretical predictions. This fundamental problem serves as a validation benchmark for CFD-DEM solvers, as it allows direct comparison with analytical solutions and experimental data. The particle motion is governed by the balance between gravitational, buoyancy, and hydrodynamic drag forces. The performance of the newly developed solver is compared against theoretical predictions based on Stokes' law and empirical drag correlations for higher Reynolds numbers, as well as against results obtained from CFDEM Coupling simulations.\par

A spherical particle with specified diameter and density is released from rest in a static fluid column. The computational domain consists of a rectangular channel with dimensions sufficient to minimize wall effects on particle motion. The particle is initially positioned at the top of the domain, and the simulation tracks its trajectory, velocity, and settling velocity as it falls under gravity. Simulations are performed using the newly developed solver, where the fluid phase is solved using the adapted pimpleFoam solver and particle dynamics are handled by the implemented DEM solver. For comparison, parallel simulations are conducted using CFDEM Coupling, where the fluid phase is solved using OpenFOAM's pimpleFoam solver and particle dynamics are handled by LIGGGHTS through the CFDEM coupling interface. The mesh resolution is chosen consistently across all solvers to ensure adequate resolution of the fluid flow around the particle, with refinement in the vicinity of the particle to capture boundary layer effects accurately. Key parameters such as particle trajectory, velocity evolution, and terminal settling velocity are extracted from all simulations for comparative analysis.\par

\begin{figure}[ht]
	\centering
	\includegraphics[width=0.5\textwidth]{Figures/Chapters/C04/single_particle_case.png}
	\caption{Geometry for simulating free sedimentation of a single spherical particle with dimensions in cm}
	\label{fig:Geometry for simulating free sedimentation of a single spherical particle with dimensions in cm}
\end{figure}

\subsection{Test Case B: DEM Pouring Heap Formation}

This test case evaluates the performance of the newly developed CFD-DEM solver in simulating the formation of a granular heap through particle pouring, with results compared against CFDEM Coupling and experimental measurements. This fundamental problem serves as a validation benchmark for DEM solvers, as it tests the accuracy of particle-particle contact models, friction coefficients, and the ability to capture realistic granular material behavior. The heap formation process involves particles being poured from a source onto a horizontal surface, where they accumulate and form a characteristic conical or asymmetric heap shape. The performance of the newly developed solver is compared against experimental measurements of heap geometry, including the angle of repose, heap height, and base diameter, which are sensitive to particle properties such as friction, restitution coefficient, and particle size distribution. Additionally, results are compared with those obtained from CFDEM Coupling simulations to assess the relative performance of the solvers.\par

Particles with specified diameter, density, and material properties are released from a hopper or point source located above a horizontal base plate. The computational domain consists of a rectangular container with dimensions sufficient to accommodate the fully formed heap without significant wall effects. Particles are introduced at a controlled rate from the source, and the simulation tracks their motion, collisions, and accumulation on the base surface. Simulations are performed using the newly developed solver, where particle dynamics including contact forces, friction, and collisions are handled by the implemented DEM solver. For comparison, parallel simulations are conducted using CFDEM Coupling, where particle dynamics are handled by LIGGGHTS through the CFDEM coupling interface. The mesh resolution and particle properties are kept consistent across all simulations to ensure a fair comparison. Key parameters monitored during the simulation include the evolution of heap geometry, particle velocity distributions, contact force networks, and the final angle of repose, which provides a direct measure of the material's frictional properties and the solver's accuracy in modeling granular mechanics. These parameters are extracted from all simulations for systematic comparative analysis.\par

\begin{figure}[ht]
	\centering
	\includegraphics[width=0.5\textwidth]{Figures/Chapters/C04/dem_heap_case.png}
	\caption{Geometry for simulating DEM pouring heap formation with dimensions in cm}
	\label{fig:Geometry for simulating DEM pouring heap formation with dimensions in cm}
\end{figure}

\subsection{Test Case C: Particle-Laden Flow in a Horizontal Channel}

To evaluate the performance of the newly developed CFD-DEM solver in handling complex particle-fluid interactions, a test case involving multiple particles in a horizontal channel flow is simulated, with results compared against OpenHFDIB-DEM. This configuration tests the solver's capability to handle particle-particle collisions, particle-wall interactions, and the coupling between fluid flow and particle motion. The OpenHFDIB-DEM solver's ability to represent irregular particle shapes using the hybrid Fictitious Domain-Immersed Boundary method \parencite{studenik_openhfdib-dem_2024,isoz_hybrid_2022} provides a benchmark for comparison, particularly relevant for sediment transport applications where natural grains exhibit non-spherical geometries.\par

The computational domain consists of a horizontal channel with specified length, width, and height. A uniform inlet velocity profile is applied, creating a fully developed turbulent flow. Spherical and non-spherical particles are introduced at the inlet, and their transport, settling, and interaction with the channel bed are monitored throughout the simulation. Simulations are performed using the newly developed solver, where the fluid phase is solved using the adapted pimpleFoam solver and particle dynamics are handled by the implemented DEM solver. For comparison, parallel simulations are conducted using OpenHFDIB-DEM, which employs the hybrid Fictitious Domain-Immersed Boundary method for particle representation. The mesh is generated using OpenFOAM utilities with consistent refinement strategies across all solvers, including appropriate refinement near the channel walls and in regions of high particle concentration. Boundary conditions include no-slip walls at the bottom and sides, a symmetry or free-slip condition at the top, and appropriate inlet-outlet conditions for the fluid phase. Particle properties, including size distribution, density, and shape parameters, are initialized identically across all simulations to match experimental or theoretical configurations. The simulation tracks particle trajectories, velocities, and fluid flow fields from all solvers, providing comprehensive data for validation against experimental measurements and for comparative analysis between the newly developed solver and OpenHFDIB-DEM.\par

\begin{figure}[ht]
	\centering
	\includegraphics[width=0.9\textwidth]{Figures/Chapters/C04/sediment_bed_sim.png}
	\caption{Geometry for simulating particle-laden flow in a horizontal channel with dimensions in cm}
	\label{fig:Geometry for simulating particle-laden flow in a horizontal channel with dimensions in cm}
\end{figure}

\section{OpenFOAM implementation of CFD-DEM Solver}

The implementation of the CFD-DEM solver in OpenFOAM involves developing a coupled numerical framework to simulate fluid phase flow interactions with particle geometry. OpenFOAM-v2312 from OpenCFD Ltd. \parencite{noauthor_openfoam_nodate} is forked from url https://develop.openfoam.com/Development/openfoam/-/blob/master/doc/Build.md on top of which the DEM method is implemented. The methodological flowchart for the implementation of the solver is given as follows:\par

\begin{figure}[ht]
	\centering
	\includegraphics[width=1.0\textwidth]{Figures/Chapters/C04/openfoam_implementation_chart.png}
	\caption{Methodology flowchart of solver implementation in OpenFOAM}
	\label{fig:Methodology flowchart of solver implementation in OpenFOAM}
\end{figure}

\subsection{Class Structure and Object-Oriented Design}

Making use of C++ object-oriented functionality, each component in the solver, from particle properties to fluid fields, is encapsulated within specific classes:\par

\begin{itemize}
	\item  aggregate\textbf{ }class: The class represents individual particles in the flow domain, with properties such as size, density, position, velocity, and mass. It includes methods for updating particle position, velocity, and geometry creation tools.\par
	\item  backgroundMesh\textbf{ }class: The class represents body custom generator to produce 3D mesh for fluid domain and body fitted immersed boundaries for particles.\par
	\item  demUpdater\textbf{ }class: The class is used to determine pressure forces from pressure field and to calculate the acceleration and velocity by solving Newton's second law of motion.\par
\end{itemize}

\subsection{Discretization and Mesh Handling}

The computational domain is discretized using a three-dimensional Cartesian grid, which adapts dynamically to accommodate the geometry of sediments within the fluid. The mesh is defined by global bounds and resolution, balancing accuracy with computational efficiency. Sediment aggregates are modeled as discrete geometries introduced into the flow domain using a distribution of sizes and orientations. The integration of sediment boundaries into the computational grid is achieved through geometry intersection techniques, which detect and resolve interactions between the sediment surfaces and mesh cells. This process updates the mesh dynamically by modifying vertices, faces, and cells to incorporate sediment as solid boundaries, ensuring accurate representation of sediment-fluid interactions within the simulation domain.\par

\subsection{Fluid Flow Computation}

The fluid phase is solved using an adapted OpenFOAM solver pimpleFoam that has been modified to incorporate particle interactions. Pressure-velocity coupling is managed using PIMPLE algorithm to ensure the stability of transient simulations. Boundary conditions, including inlet velocity profiles and no-slip walls, are implemented to accurately model realistic fluid flows within the domain.\par

\subsection{Discrete Element Method (DEM) Solver}

The DEM solver performs force calculation which includes pressure and gravitational forces acting on particles and computes acceleration, velocity, and position of the particle for the next time-step using Euler time integration scheme.\par

\subsection{Coupling CFD and DEM Phases}

The coupling between the pimple and DEM solvers is achieved through a one-way exchange of momentum and energy. The fluid phase exerts pressure force on particles, calculated based on pressure field obtained by the solution of pimple solver. At each time-step, the fluid solver is executed first, followed by the DEM solver. The iteration loop continues until convergence, or the specified time is reached.\par

\subsection{Solver Testing and Result Evaluation}

To ensure the solver's functionality, the simplest case is solved using the solver i.e. free sedimentation of a single particle. The solver outputs particle positions, velocities, and fluid fields at each time step, recorded for post-processing. The results are saved in ASCII formats for analysis and visualization in the software ParaView.\par
\textbf{Statistical analysis:} Additional post-processing routines compute path lines, forces, and stress distributions, providing insights into particle transport and interaction effects.\par
